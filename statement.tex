% Created 2017-01-16 Mon 19:22
% Intended LaTeX compiler: xelatex
\documentclass[11pt]{article}
\usepackage{amsmath}
\usepackage{graphicx}
\usepackage{longtable}
\usepackage{float}
\usepackage{fontspec}
\usepackage{unicode-math}
\usepackage{xunicode}
\usepackage[backend=biber,date=year,defernumbers=true,doi=false,isbn=false,style=nature,url=false]{biblatex}
\usepackage{tikz}
\usepackage[xetex]{hyperref}
\usepackage{rotating}
\usepackage{parskip}
\usepackage[letterpaper,margin=1in]{geometry}
\renewcommand\cite{\supercite}
\addbibresource{/home/asarkar/research/mit/reading/reading.bib}
\setmainfont[Extension=.otf,UprightFont=*-regular,BoldFont=*-bold,ItalicFont=*-italic,BoldItalicFont=*-bolditalic]{texgyrepagella}
\setmathfont[bold-style=TeX]{texgyrepagella-math.otf}
\date{}
\title{Abhishek Sarkar research statement}
\hypersetup{
 pdfauthor={Abhishek Sarkar},
 pdftitle={Abhishek Sarkar research statement},
 pdfkeywords={},
 pdfsubject={},
 pdfcreator={Emacs 25.1.1 (Org mode 9.0.3)}, 
 pdflang={English}}
\begin{document}

\begin{center}
  \large\bfseries\makeatletter\@title\makeatother
\end{center}

My career goal is to address one of the most important questions in human
genetics: the mechanistic basis by which genetic variation causes human
diseases. I have been an active member of the field over the last 10 years, and
I have watched it transform from a field where genome-wide association studies
(GWAS) of hundreds of individuals would identify a handful of genetic loci and
make scientific headlines, to an incredibly collaborative field which has
discovered tens of thousands of genetic associations over hundreds of diseases,
and is moving towards studying millions of individuals.

However, I have also felt the disappointment in the field that hundreds of new
genetic loci have only opened up hundreds of new questions, of which only a
handful have been answered by extensive experimentation, rather than leading
directly to mechanistic insights and new therapeutics. The vast majority of
disease loci fall in non-coding regions and elucidating their functional role
in human disease is an unsolved challenge. Even for protein-coding
associations, identifying the causal variant and understanding where and how
the disrupted gene acts is still a challenge. But for non-coding associations,
we don't know the target gene, let alone the causal variant, the cell type in
which it acts, the regulators that control it, the effect on the target gene,
or the intermediate phenotypes which mediate its effect on the observed
phenotype. Moreover, the field is increasingly recognizing that many human
diseases are highly polygenic, involving many yet uncharacterized loci which
even the largest meta-analyses to date are under-powered to detect. However,
this observation only makes interpreting the role of genetic variation in human
disease more difficult, as smaller individual effects become detectable only in
aggregate over biological pathways.

My goal is to address these challenges on two fronts. First, I aim to gain
insight into the non-coding genome, the role it plays in epigenetic and
transcriptional regulation, and ultimately how non-coding genetic variation
contributes to human disease. Second, I aim to develop new computational
methods to probe these biological questions by jointly analyzing diverse data
such as GWAS, epigenomic profiles, regulatory circuits, gene expression, and
intermediate phenotypes in order to elucidate the mechanistic basis and
gene-regulatory architecture of human disease.

With the Roadmap Epigenomics Consortium, I developed methods to yield new
biological insights into the mechanistic role of non-coding variation in
disease. My methods combined summary statistics for eight diseases spanning
psychiatric, autoimmune, and metabolic disorders with the most comprehensive
set of epigenomic maps of regulatory regions across 127 human tissues and cell
types, regulatory motifs of 651 transcription factors, and thousands of gene
pathways. I went beyond state-of-the-art methods to identify specific genetic
variants, the cellular context in which they were functional, the upstream
regulator whose binding they disrupt, and the downstream gene whose expression
they regulate. My analysis revealed the role of immune-specific regulatory
regions in psychiatric disorders, identified shared master regulators between
psychiatric and auto-immune disorders, highlighted the importance of
constitutive enhancers active in all cell types, and implicated specific
biological processes for experimental followup.

I have applied my methods to other diverse studies in collaboration with GWAS
consortia and industry partners. With the Reproductive Genetics Consortium, I
applied my methods to studies of age to menarche, revealing new genes
previously not implicated in the phenotype, and to age at menopause, revealing
a role for immune response in the phenotype previously studied but rejected by
the field. I am currently working in collaboration with the GWAS group of the
ENCODE Consortium, the GWAS of Sequencing of Alcohol and Nicotine use
consortium, the Psychiatric Genetics Consortium (PGC), and Pfizer to apply my
methods to study an even larger range of phenotypes.

These studies have laid the foundation for research directions and projects I
am interested in pursuing. First, I want to understand the genetic architecture
of complex traits, especially human disease. Specifically, I aim to develop
methods which can estimate the number of causal variants, give insight into
where those causal variants can be found in the human genome, and motivate new
designs for future human genetic studies. I am currently making progress along
these lines, developing a new method to combine GWAS data with regulatory
regions predicted using epigenomic annotations. My immediate goal is to
distinguish between non-infinitesimal genetic architectures where regulatory
regions harbor more causal variants for disease from architectures where causal
variants are distributed uniformly. In order to do so, I have extended
state-of-the-art large-scale Bayesian sparse regression models to identify and
avoid degenerate solutions, scale to biobank-scale data sets of hundreds of
thousands of individuals, and validate their robustness to violation of the
model assumptions. 

Currently, one research question I aim to investigate using my method is
explaining recent contradictory results on the role of immune cell types in
schizophrenia. A number of methods have been proposed to study enrichment of
disease associations in regulatory regions; however, only some of these methods
find enrichment in immune-specific regulatory regions. Using a model-based
approach, I aim to infer the gene regulatory architecture of schizophrenia in
collaboration with the PGC which could explain why different methods which
study fundamentally different quantities (the number of associations
overlapping regulatory elements versus the proportion of phenotype variance
they explain) give contradictory results.

Second, I want to build mechanistic understanding of the transcriptional
regulatory network incorporating additional information such as epigenomic
profiles and chromatin structure. One of the key challenges I aim to address is
understanding how expression quantitative trait loci (eQTLs) vary across
tissues, and explaining those differences using epigenomic and regulatory
network contexts. I am currently making some progress along these lines in
joint work with Yongjin Park at MIT, developing a new method to jointly map
eQTLs in multiple tissues by factorizing to deconvolve SNP-specific effects and
tissue-specific effects. The fundamental insight which I aim to capture is that
the effect of an eQTL on target gene expression can be deconvolved into a
genetic effect on a regulatory motif (changing the binding affinity for an
upstream transcription factor), a tissue effect on local epigenetic state
(changing chromatin accessibility or histone modification to create or
deactivate enhancer elements), and a tissue effect on the upstream regulator
(changing the gene expression of the upstream regulator).

There is a pressing need to distinguish between different modes of
tissue-specific gene regulatory action in order to accurately predict the role
of genetic variation on transcriptional regulation, and through regulation on
disease phenotypes. However, in order to truly deconvolve these effects, we
need to build more complex models incorporating not only genotype and gene
expression profiles, but also sequence-based regulatory potential (especially
regulatory motifs) and epigenomic profiles in matched or similar cell types.
These models will require not only existing large-scale data sets, but also
more comprehensive data in relevant cell types, additional developmental time
points, and across multiple individuals to more exactly pinpoint the tissue and
mechanism of action and new algorithms to efficiently perform the statistical
inference.

Third, I want to study the impact of inter-individual variation in intermediate
phenotypes such as histone modification, chromatin accessibility, DNA
methylation, and gene expression on human disease. I have already made some
progress in this direction. In joint work with Kunal Bhutani at UCSD, I
contributed to a new method for error modeling in transcriptome-wide
association studies (TWAS). TWAS exploits reference gene expression profiles,
generated by large-scale efforts such as the Gene Tissue Expression Project, to
impute unobserved gene expression into large-scale GWAS cohorts and associate
inter-individual changes in gene expression with disease, directly implicating
new genes. In collaboration with Yongjin Park at MIT, I am contributing to new
efficient approximate methods to incorporate epigenetic regulation through
methylation in TWAS, and more generally to model multiple levels of mediation
through intermediate phenotypes. Moving forward, the key area I am interested
in is generating or imputing new data to jointly study case-control differences
in multiple gene regulatory phenotypes and other disease endophenotypes.

Fourth, I want to address a fundamental shortcoming in state-of-the-art methods
to analyze genomic, epigenomic, transcriptomic, and phenotypic data: accounting
for error and uncertainty in predictions of regulatory elements and regulatory
networks. Accounting for uncertainty in a principled manner necessitates
building larger integrative models combining diverse data types, new methods to
relax assumptions, and new implementations to efficiently analyze the data. The
field is increasingly recognizing the value of epigenomic annotations to
interpret the role of genetic variation in the context of transcriptional
regulation; however, our use of these annotations is dependent on computational
models with unknown error rates and predictions without corresponding error
estimates. To address these issues, I aim to learn deep generative models for
epigenomic annotations, and incorporate these generative models as subunits
into larger generative models for disease. In my current work, I have worked on
such a generative regression model for disease which assumes regulatory
annotation data is observed, but can be easily extended to sample regulatory
annotations from the posterior distribution of a separately trained generative
model.
\end{document}
