\documentclass{letter}
\usepackage[letterpaper,textwidth=6.5in,textheight=9in]{geometry}
\setlength\longindentation{0pt}
\usepackage{fontspec}
\setmainfont{texgyretermes-regular.otf}
\signature{Abhishek Sarkar}

\begin{document}
  \opening{To whom it may concern,}

  I am enclosing my application for the position of Scientist in Statistical
  Genetics at 23andMe.

  My career goal is to address one of the most important questions in human
  genetics: the mechanistic basis by which genetic variation causes human
  diseases. I have been an active member of the field over the last 6 years, and
  I have watched it transform from a field where genome-wide association studies
  (GWAS) of hundreds of individuals would identify a handful of genetic loci and
  make scientific headlines, to an incredibly collaborative field which has
  discovered tens of thousands of genetic associations over hundreds of diseases,
  and is moving towards studying millions of individuals. However, I have also
  felt the disappointment in the field that hundreds of new genetic loci have
  only opened up hundreds of new questions, of which only a handful have been
  answered by extensive experimentation, rather than leading directly to
  mechanistic insights and new therapeutics.

  I aim to address these challenges on two fronts. First, I aim to gain insight
  into epigenetic and transcriptional regulation and ultimately how non-coding
  genetic variation contributes to human disease. Second, I aim to develop new
  computational methods to probe these biological questions by jointly
  analyzing diverse data at large scale such as GWAS, epigenomic profiles,
  regulatory circuits, gene expression, and intermediate phenotypes.

  During the course of my PhD, I developed methods as part of the Roadmap
  Epigenomics Consortium to yield new biological insights into the mechanistic
  role of non-coding variation in disease. My methods combined summary statistics
  for eight diseases spanning psychiatric, autoimmune, and metabolic disorders
  with the most comprehensive set of epigenomic maps of regulatory regions across
  127 human tissues and cell types, regulatory motifs of 651 transcription
  factors, and thousands of gene pathways. I went beyond state-of-the-art methods
  to identify specific genetic variants, the cellular context in which they were
  functional, the upstream regulator whose binding they disrupt, and the
  downstream gene whose expression they regulate.

  This work has laid the foundation for research directions I am interested in
  pursuing.

  \begin{itemize}
  \item I want to understand the genetic architecture of human disease: how many
    causal variants there are, and where they fall in the genome. I am currently
    making progress along these lines, developing a new efficient approximate
    Bayesian inference algorithm for GWAS/biobank-scale data.
  \item I want to build mechanistic understanding of transcriptional regulation
    to integrate with human genetic variation data. I have made some progress
    along these lines in recent joint work on multi-tissue eQTL mapping and
    transcriptome-wide association studies (TWAS).
  \item I want to study the impact of inter-individual variation in
    intermediate molecular phenotypes and endophenotypes, as well as on human
    disease. One approach we can use to make progress on this front is
    mediation analysis, using ideas from Mendelian randomization and TWAS.
  \end{itemize}

  23andMe has one of the richest and largest datasets to answer these biological
  questions, and I believe that the company will need to make advances along
  these directions in order to further its goals of translating its datasets into
  novel therapeutic targets. I believe that I am an ideal candidate to work on
  these research directions due to my experience in integrating diverse data
  types to gain insight into the genetics of human disease and my experience in
  building complex models to gain the most specific biological insights.

  \closing{Thank you for your consideration,}
\end{letter}

\end{document}
