\documentclass{article}
\usepackage{fontspec}
\defaultfontfeatures{Mapping=tex-text, Scale=MatchLowercase}
\setmainfont{Minion Pro}
\setmonofont{Consolas}
\usepackage[letterpaper, textwidth=5.5in, textheight=9in]{geometry}
\usepackage{hyperref}
\usepackage{sectsty}
\sectionfont{\mdseries\large}
\usepackage{enumitem}
% Lists with hanging indents
\newlist{hdesc}{description}{1}
\setlist[hdesc]{format=\small\mdseries\scshape, align=right,
  labelindent=-4.45em, itemindent=0pt, labelwidth=4em, leftmargin=!,
  itemsep=0pt}
\newlist{henum}{enumerate}{1}
\setlist[henum]{label=\arabic*., labelwidth=4em, labelindent=-4.45em,
  itemindent=0pt, leftmargin=!, itemsep=0pt}

\setlength{\parindent}{0pt}
\pagestyle{empty}

\begin{document}
{\LARGE Abhishek Sarkar\vspace{1em}}\\
Stata Center 32-D526\\
32 Vassar Street\\
Cambridge, MA 02139\vspace{1em}

(919) 259-1733

\begin{hdesc}
\item[email] \url{aksarkar@mit.edu}
\item[url] \url{http://mit.edu/aksarkar}
\end{hdesc}

\section*{Research interests}
Complex traits, regulatory genomics, epigenomics, Bayesian inference

\section*{Education}
\begin{hdesc}
\item[2017] Ph.D.\ Computer Science, Massachusetts Institute of Technology
  (expected)
\item[2013] M.S.\ Computer Science, Massachusetts Institute of Technology
\item[2011] B.S.\ Computer Science with Highest Honors, University of North
  Carolina at Chapel Hill
\end{hdesc}

\section*{Research experience}
\begin{hdesc}[itemsep=1em]
\item[2011--] \textbf{Massachusetts Institute of Technology}\\
Research Assistant\\
Computational Biology Group, Manolis Kellis (PI)\\
Computer Science and Artificial Intelligence Lab
\end{hdesc}

\section*{Teaching experience}
\begin{hdesc}[itemsep=1em]
\item[2014] Teaching assistant, ``Computational Biology: Genomes, Networks, Evolution'' (fall)
\end{hdesc}

\section*{Pre-prints}
\begin{henum}
\item Felix Day, \ldots, \textbf{Abhishek K.\ Sarkar}, et al. ``Genomic
  analyses for age at menarche identify 389 independent signals and indicate
  BMI-independent effects of puberty timing on cancer susceptibility.''
  \emph{BioRxiv}. (2016) \emph{Author 9/215}
\item \textbf{Abhishek K.\ Sarkar}, Lucas D.\ Ward, Manolis Kellis.
  ``Functional enrichments of disease variants across thousands of independent
  loci in eight diseases.'' \emph{BioRxiv}. (2016)
  \href{http://biorxiv.org/content/early/2016/04/11/048066}{\texttt{doi:10.1101/048066}}
\item Yaping Liu, \textbf{Abhishek Sarkar}, Manolis Kellis. ``Evidence of a
  recombination rate valley in human regulatory domains.'' \emph{BioRxiv}.
  (2016) \href{http://biorxiv.org/content/early/2016/04/15/048827}{\texttt{doi:10.1101/048827}}
\end{henum}

\section*{Peer-reviewed publications}
\begin{henum}
\item Roadmap Epigenomics Consortium et al. ``Integrative analysis of 111
  reference human epigenomes.'' \emph{Nature}, 518(7539), 317–330 (2015).
  \emph{Integrative analysis lead (equal contributor)}.
\end{henum}

\section*{Presentations}
\begin{henum}
\item \textbf{Abhishek Sarkar}, Yongjin Park, Manolis Kellis. ``Dissecting the
  non-infinitesimal architecture of complex traits using group spike-and-slab
  priors'' (contributed talk). Workshop on Machine Learning in Computational
  Biology, Thirtieth Annual Conference on Neural Information Processing
  Systems, Barcelona, Spain. 2016.
\item \textbf{Abhishek Sarkar}, Yongjin Park, Manolis Kellis. ``Dissecting the
  non-infinitesimal architecture of complex traits'' (poster).
  68\textsuperscript{th} meeting of the American Society of Human Genetics,
  Vancouver, Canada. 2016.
\item \textbf{Abhishek Sarkar}, Luke Ward, Manolis Kellis. ``Functional
  enrichments of disease variants across thousands of independent loci in eight
  diseases.'' (talk). Leena Peltonen School of Human Genomics, Wellcome Trust
  Sanger Institute, Hinxton, Cambridge, UK. 2016.
\item Yongjin Park, \textbf{Abhishek Sarkar}, Nick Mancuso, Alexander Gusev,
  Bogdan Pasaniuc, Manolis Kellis. ``Computational discovery of epigenetic
  mediators in Alzheimer’s disease from imputed methyome-wide association
  statistics'' (poster). The Biology of Genomes, Cold Spring Harbor, NY, USA.
  2016.
\item Kunal Bhutani$^*$, \textbf{Abhishek Sarkar}$^*$, Yongjin Park, Manolis
  Kellis, Nicholas Schork. ``Propagating uncertainty of predicted expression in
  transcriptome-wide association studies'' (poster). The Biology of Genomes,
  Cold Spring Harbor, NY, USA. 2016. \emph{$^*$Equal contribution}
\item \textbf{Abhishek Sarkar}, Lucas D.\ Ward, Manolis Kellis.
  ``Regulatory annotations implicate thousands of independent loci'' (poster).
  67\textsuperscript{th} meeting of the American Society of Human Genetics,
  Baltimore, MD, USA. 2015.
\item \textbf{Abhishek K.\ Sarkar}, Lucas D.\ Ward, Manolis
  Kellis. ``Genome-wide enrichments for regulatory regions across thousands of
  unlinked disease-associated variants'' (poster). 65\textsuperscript{th}
  meeting of the American Society of Human Genetics, Boston, MA, USA. 2013.
\item Vineeta Agarwala, \textbf{Abhishek Sarkar}, Kyle Gaulton. ``Using the
  Epigenome Roadmap data to analyze genetic studies of Type 2 Diabetes''
  (workshop talk). 65\textsuperscript{th} meeting of the American Society of
  Human Genetics, Boston, MA, USA. 2013.
\item \textbf{Abhishek K.\ Sarkar}, Lucas D.\ Ward, Manolis
  Kellis. ``Systematically interpreting GWAS using regions from Roadmap''
  (poster). Epigenomics: A Roadmap to the Living Genome, Boston, MA, USA. 2013.
\item Lucas Ward, \textbf{Abhishek Sarkar}, Manolis Kellis. ``Global
  contributions of regulatory elements to disease risk and evolutionary fitness
  in the human population'' (poster). 5\textsuperscript{th} annual RECOMB
  Conference on Regulatory and Systems Genomics, San Francisco, CA, USA. 2012.
\item \textbf{Abhishek Sarkar}. ``Functional GWAS enrichments across tens of
  thousands of enhancer elements'' (talk). Epigenomics Seminar Series, Broad
  Institute, Cambridge, MA, USA. 2011.
\end{henum}

\section*{Honors}
\begin{hdesc}
\item[2016] Accepted to Leena Peltonen School of Human Genomics
\item[2011] Awarded NSF Graduate Research Fellowship
\item[2011] Inducted into Phi Beta Kappa honors fraternity
\end{hdesc}

\end{document}
