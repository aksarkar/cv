\documentclass[11pt]{letter}
\usepackage[letterpaper,textwidth=6.5in,textheight=9in]{geometry}
\setlength\longindentation{0pt}
\usepackage{fontspec}
\setmainfont{texgyretermes-regular.otf}
\signature{Abhishek Sarkar}

\begin{document}{}
  \opening{To whom it may concern,}

  I am enclosing my application for the position of Computational Biologist at
  Pfizer.

  My career goal is to address one of the most important questions in human
  genetics: the mechanistic basis by which genetic variation causes human
  diseases. I have been an active member of the field over the last 6 years,
  and I have watched it transform from a field where genome-wide association
  studies (GWAS) of hundreds of individuals would identify a handful of genetic
  loci and make scientific headlines, to an incredibly collaborative field
  which has discovered tens of thousands of genetic associations over hundreds
  of diseases, and is moving towards studying millions of individuals. However,
  I have also felt the disappointment in the field that hundreds of new genetic
  loci have only opened up hundreds of new questions, of which only a handful
  have been answered by extensive experimentation, rather than leading directly
  to mechanistic insights and new therapeutics.

  During the course of my PhD, I developed methods as part of the Roadmap
  Epigenomics Consortium to yield new biological insights into the mechanistic
  role of non-coding variation in disease. My methods combined summary
  statistics for eight diseases spanning psychiatric, autoimmune, and metabolic
  disorders with the most comprehensive set of epigenomic maps of regulatory
  regions across 127 human tissues and cell types, regulatory motifs of 651
  transcription factors, and thousands of gene pathways. I went beyond
  state-of-the-art methods to identify specific genetic variants, the cellular
  context in which they were functional, the upstream regulator whose binding
  they disrupt, and the downstream gene whose expression they regulate.

  I have additionally collaborated on a number of Bayesian models of genetics,
  gene expression, and disease. These models have allowed us to directly
  analyze the role of disruption of gene regulation and aberrant gene
  expression on phenotypes of interest to Pfizer, such as Crohn's disease. They
  have also spurred the development of generic machine learning algorithms
  which we can use to rapidly design, implement, and test complex models
  integrating diverse data types.

  This work has laid the foundation for my interest in translating
  computational predictions about gene regulatory mechanisms causing disease
  into novel therapeutic targets. I believe that we need to make advances in
  two directions in order to achieve this goal. First, we need to gain insight
  into epigenetic and transcriptional regulation and ultimately how non-coding
  genetic variation contributes to human disease. Second, we need to develop
  new computational methods to probe these biological questions by jointly
  analyzing diverse data at large scale such as GWAS, epigenomic profiles,
  regulatory circuits, gene expression, and intermediate phenotypes.

  My research experience in these areas will be invaluable in pursuing the goal
  of identifying novel therapeutic targets based on genetic and molecular
  evidence, making me an ideal candidate for the position.

  \closing{Thank you for your consideration,}
\end{letter}

\end{document}
